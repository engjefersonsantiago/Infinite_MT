The P4 language and modern programmable dataplanes have redrawn the networking landscape by allowing full data path programming in SDN environments.
P4 offers an explicit imperative match-action-based programming model, which is the main processing abstraction in programmable dataplanes.
However, modern programmable dataplanes lack the memory capacity to implement large match tables.
Recent research has suggested to use heterogeneous programmable dataplanes (HDPs) to increase the memory capacity. 
Such an HDP is made of different programmable dataplane devices (PDDs).
Each of these devices has its own memory capacity and processing capabilities, in a way that the most memory abundant device has the lowest performance and vice-versa.
Hence, the bandwidth supported by an HDP is limited by the slowest PDD.

To address this issue, this work presents a cache hierarchy scheme for HPDs that allows to implement large match tables, while supporting a high packet throughput. 
We start our analysis by characterizing a recent data center trace.
Then, following our observations, we derive the caching premises for match-action caching.
We develop an open-source simulator to evaluate different caching schemes.
Finally, our simulations suggest that a two-level cache hierarchy that employs a replacement policy combining random eviction with heuristic promotion can achieve a hit ratio that approaches the theoretical maximum, with a relatively small cache and low implementation costs.