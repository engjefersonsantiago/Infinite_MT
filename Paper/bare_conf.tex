
%% bare_conf.tex
%% V1.4b
%% 2015/08/26
%% by Michael Shell
%% See:
%% http://www.michaelshell.org/
%% for current contact information.
%%
%% This is a skeleton file demonstrating the use of IEEEtran.cls
%% (requires IEEEtran.cls version 1.8b or later) with an IEEE
%% conference paper.
%%
%% Support sites:
%% http://www.michaelshell.org/tex/ieeetran/
%% http://www.ctan.org/pkg/ieeetran
%% and
%% http://www.ieee.org/

%%*************************************************************************
%% Legal Notice:
%% This code is offered as-is without any warranty either expressed or
%% implied; without even the implied warranty of MERCHANTABILITY or
%% FITNESS FOR A PARTICULAR PURPOSE! 
%% User assumes all risk.
%% In no event shall the IEEE or any contributor to this code be liable for
%% any damages or losses, including, but not limited to, incidental,
%% consequential, or any other damages, resulting from the use or misuse
%% of any information contained here.
%%
%% All comments are the opinions of their respective authors and are not
%% necessarily endorsed by the IEEE.
%%
%% This work is distributed under the LaTeX Project Public License (LPPL)
%% ( http://www.latex-project.org/ ) version 1.3, and may be freely used,
%% distributed and modified. A copy of the LPPL, version 1.3, is included
%% in the base LaTeX documentation of all distributions of LaTeX released
%% 2003/12/01 or later.
%% Retain all contribution notices and credits.
%% ** Modified files should be clearly indicated as such, including  **
%% ** renaming them and changing author support contact information. **
%%*************************************************************************


% *** Authors should verify (and, if needed, correct) their LaTeX system  ***
% *** with the testflow diagnostic prior to trusting their LaTeX platform ***
% *** with production work. The IEEE's font choices and paper sizes can   ***
% *** trigger bugs that do not appear when using other class files.       ***                          ***
% The testflow support page is at:
% http://www.michaelshell.org/tex/testflow/



\documentclass[conference,dvipsnames,10pt]{IEEEtran}
% Some Computer Society conferences also require the compsoc mode option,
% but others use the standard conference format.
%
% If IEEEtran.cls has not been installed into the LaTeX system files,
% manually specify the path to it like:
% \documentclass[conference]{../sty/IEEEtran}


%\makeatletter
%\def\bstctlcite{\@ifnextchar[{\@bstctlcite}{\@bstctlcite[@auxout]}}
%\def\@bstctlcite[#1]#2{\@bsphack
% \@for\@citeb:=#2\do{%
%   \edef\@citeb{\expandafter\@firstofone\@citeb}%
%   \if@filesw\immediate\write\csname #1\endcsname{\string\citation{\@citeb}}\fi}%
% \@esphack}
%\makeatother



% Some very useful LaTeX packages include:
% (uncomment the ones you want to load)


% *** MISC UTILITY PACKAGES ***
%
%\usepackage{ifpdf}
% Heiko Oberdiek's ifpdf.sty is very useful if you need conditional
% compilation based on whether the output is pdf or dvi.
% usage:
% \ifpdf
%   % pdf code
% \else
%   % dvi code
% \fi
% The latest version of ifpdf.sty can be obtained from:
% http://www.ctan.org/pkg/ifpdf
% Also, note that IEEEtran.cls V1.7 and later provides a builtin
% \ifCLASSINFOpdf conditional that works the same way.
% When switching from latex to pdflatex and vice-versa, the compiler may
% have to be run twice to clear warning/error messages.






% *** CITATION PACKAGES ***
%
\usepackage{cite}
% cite.sty was written by Donald Arseneau
% V1.6 and later of IEEEtran pre-defines the format of the cite.sty package
% \cite{} output to follow that of the IEEE. Loading the cite package will
% result in citation numbers being automatically sorted and properly
% "compressed/ranged". e.g., [1], [9], [2], [7], [5], [6] without using
% cite.sty will become [1], [2], [5]--[7], [9] using cite.sty. cite.sty's
% \cite will automatically add leading space, if needed. Use cite.sty's
% noadjust option (cite.sty V3.8 and later) if you want to turn this off
% such as if a citation ever needs to be enclosed in parenthesis.
% cite.sty is already installed on most LaTeX systems. Be sure and use
% version 5.0 (2009-03-20) and later if using hyperref.sty.
% The latest version can be obtained at:
% http://www.ctan.org/pkg/cite
% The documentation is contained in the cite.sty file itself.






% *** GRAPHICS RELATED PACKAGES ***
%
\ifCLASSINFOpdf
  % \usepackage[pdftex]{graphicx}
  % declare the path(s) where your graphic files are
  % \graphicspath{{../pdf/}{../jpeg/}}
  % and their extensions so you won't have to specify these with
  % every instance of \includegraphics
  % \DeclareGraphicsExtensions{.pdf,.jpeg,.png}
\else
  % or other class option (dvipsone, dvipdf, if not using dvips). graphicx
  % will default to the driver specified in the system graphics.cfg if no
  % driver is specified.
  % \usepackage[dvips]{graphicx}
  % declare the path(s) where your graphic files are
  % \graphicspath{{../eps/}}
  % and their extensions so you won't have to specify these with
  % every instance of \includegraphics
  % \DeclareGraphicsExtensions{.eps}
\fi
% graphicx was written by David Carlisle and Sebastian Rahtz. It is
% required if you want graphics, photos, etc. graphicx.sty is already
% installed on most LaTeX systems. The latest version and documentation
% can be obtained at: 
% http://www.ctan.org/pkg/graphicx
% Another good source of documentation is "Using Imported Graphics in
% LaTeX2e" by Keith Reckdahl which can be found at:
% http://www.ctan.org/pkg/epslatex
%
% latex, and pdflatex in dvi mode, support graphics in encapsulated
% postscript (.eps) format. pdflatex in pdf mode supports graphics
% in .pdf, .jpeg, .png and .mps (metapost) formats. Users should ensure
% that all non-photo figures use a vector format (.eps, .pdf, .mps) and
% not a bitmapped formats (.jpeg, .png). The IEEE frowns on bitmapped formats
% which can result in "jaggedy"/blurry rendering of lines and letters as
% well as large increases in file sizes.
%
% You can find documentation about the pdfTeX application at:
% http://www.tug.org/applications/pdftex





% *** MATH PACKAGES ***
%
%\usepackage{amsmath}
% A popular package from the American Mathematical Society that provides
% many useful and powerful commands for dealing with mathematics.
%
% Note that the amsmath package sets \interdisplaylinepenalty to 10000
% thus preventing page breaks from occurring within multiline equations. Use:
%\interdisplaylinepenalty=2500
% after loading amsmath to restore such page breaks as IEEEtran.cls normally
% does. amsmath.sty is already installed on most LaTeX systems. The latest
% version and documentation can be obtained at:
% http://www.ctan.org/pkg/amsmath





% *** SPECIALIZED LIST PACKAGES ***
%
%\usepackage{algorithmic}
% algorithmic.sty was written by Peter Williams and Rogerio Brito.
% This package provides an algorithmic environment fo describing algorithms.
% You can use the algorithmic environment in-text or within a figure
% environment to provide for a floating algorithm. Do NOT use the algorithm
% floating environment provided by algorithm.sty (by the same authors) or
% algorithm2e.sty (by Christophe Fiorio) as the IEEE does not use dedicated
% algorithm float types and packages that provide these will not provide
% correct IEEE style captions. The latest version and documentation of
% algorithmic.sty can be obtained at:
% http://www.ctan.org/pkg/algorithms
% Also of interest may be the (relatively newer and more customizable)
% algorithmicx.sty package by Szasz Janos:
% http://www.ctan.org/pkg/algorithmicx




% *** ALIGNMENT PACKAGES ***
%
%\usepackage{array}
% Frank Mittelbach's and David Carlisle's array.sty patches and improves
% the standard LaTeX2e array and tabular environments to provide better
% appearance and additional user controls. As the default LaTeX2e table
% generation code is lacking to the point of almost being broken with
% respect to the quality of the end results, all users are strongly
% advised to use an enhanced (at the very least that provided by array.sty)
% set of table tools. array.sty is already installed on most systems. The
% latest version and documentation can be obtained at:
% http://www.ctan.org/pkg/array


% IEEEtran contains the IEEEeqnarray family of commands that can be used to
% generate multiline equations as well as matrices, tables, etc., of high
% quality.




% *** SUBFIGURE PACKAGES ***
%\ifCLASSOPTIONcompsoc
%  \usepackage[caption=false,font=normalsize,labelfont=sf,textfont=sf]{subfig}
%\else
%  \usepackage[caption=false,font=footnotesize]{subfig}
%\fi
% subfig.sty, written by Steven Douglas Cochran, is the modern replacement
% for subfigure.sty, the latter of which is no longer maintained and is
% incompatible with some LaTeX packages including fixltx2e. However,
% subfig.sty requires and automatically loads Axel Sommerfeldt's caption.sty
% which will override IEEEtran.cls' handling of captions and this will result
% in non-IEEE style figure/table captions. To prevent this problem, be sure
% and invoke subfig.sty's "caption=false" package option (available since
% subfig.sty version 1.3, 2005/06/28) as this is will preserve IEEEtran.cls
% handling of captions.
% Note that the Computer Society format requires a larger sans serif font
% than the serif footnote size font used in traditional IEEE formatting
% and thus the need to invoke different subfig.sty package options depending
% on whether compsoc mode has been enabled.
%
% The latest version and documentation of subfig.sty can be obtained at:
% http://www.ctan.org/pkg/subfig




% *** FLOAT PACKAGES ***
%
%\usepackage{fixltx2e}
% fixltx2e, the successor to the earlier fix2col.sty, was written by
% Frank Mittelbach and David Carlisle. This package corrects a few problems
% in the LaTeX2e kernel, the most notable of which is that in current
% LaTeX2e releases, the ordering of single and double column floats is not
% guaranteed to be preserved. Thus, an unpatched LaTeX2e can allow a
% single column figure to be placed prior to an earlier double column
% figure.
% Be aware that LaTeX2e kernels dated 2015 and later have fixltx2e.sty's
% corrections already built into the system in which case a warning will
% be issued if an attempt is made to load fixltx2e.sty as it is no longer
% needed.
% The latest version and documentation can be found at:
% http://www.ctan.org/pkg/fixltx2e


%\usepackage{stfloats}
% stfloats.sty was written by Sigitas Tolusis. This package gives LaTeX2e
% the ability to do double column floats at the bottom of the page as well
% as the top. (e.g., "\begin{figure*}[!b]" is not normally possible in
% LaTeX2e). It also provides a command:
%\fnbelowfloat
% to enable the placement of footnotes below bottom floats (the standard
% LaTeX2e kernel puts them above bottom floats). This is an invasive package
% which rewrites many portions of the LaTeX2e float routines. It may not work
% with other packages that modify the LaTeX2e float routines. The latest
% version and documentation can be obtained at:
% http://www.ctan.org/pkg/stfloats
% Do not use the stfloats baselinefloat ability as the IEEE does not allow
% \baselineskip to stretch. Authors submitting work to the IEEE should note
% that the IEEE rarely uses double column equations and that authors should try
% to avoid such use. Do not be tempted to use the cuted.sty or midfloat.sty
% packages (also by Sigitas Tolusis) as the IEEE does not format its papers in
% such ways.
% Do not attempt to use stfloats with fixltx2e as they are incompatible.
% Instead, use Morten Hogholm'a dblfloatfix which combines the features
% of both fixltx2e and stfloats:
%
% \usepackage{dblfloatfix}
% The latest version can be found at:
% http://www.ctan.org/pkg/dblfloatfix




% *** PDF, URL AND HYPERLINK PACKAGES ***
%
%\usepackage{url}
% url.sty was written by Donald Arseneau. It provides better support for
% handling and breaking URLs. url.sty is already installed on most LaTeX
% systems. The latest version and documentation can be obtained at:
% http://www.ctan.org/pkg/url
% Basically, \url{my_url_here}.




% *** Do not adjust lengths that control margins, column widths, etc. ***
% *** Do not use packages that alter fonts (such as pslatex).         ***
% There should be no need to do such things with IEEEtran.cls V1.6 and later.
% (Unless specifically asked to do so by the journal or conference you plan
% to submit to, of course. )


% correct bad hyphenation here
\hyphenation{op-tical net-works semi-conduc-tor}


\usepackage{booktabs} % For formal tables
%\usepackage{siunitx}
\usepackage[detect-weight=true, load-configurations=binary]{siunitx}
\DeclareSIUnit{\nothing}{\relax}
\usepackage{float}
\usepackage{enumitem}

\usepackage{balance}
\usepackage{subfig}
\usepackage{caption}
%\usepackage{subcaption}
\usepackage{pgfplots}
\usepackage{pgf}
\usepackage{pgfplotstable}
\usepackage{tikz}
\usetikzlibrary{patterns}
\usetikzlibrary{snakes,arrows,shapes}
\usepackage{tikz,tkz-tab}%package pour les tableaux de variations et de signes%
%\usetikzlibrary{babel}
%\usepackage{modules/tikzNetwork}
\usepackage{circuitikz}
\usetikzlibrary{positioning, automata, graphs, trees, fit, arrows, shapes}
\usetikzlibrary{backgrounds,patterns,matrix,calc,shadows,plotmarks, circuits.logic.US}
\usetikzlibrary{decorations}
\usetikzlibrary{positioning, automata, graphs, trees, fit, arrows.meta, shapes}
\usetikzlibrary{backgrounds,patterns,matrix,calc,shadows,plotmarks, circuits.logic.US}
\usetikzlibrary{decorations}

\usepackage{textcomp}
%\usepackage{algorithm}
%\usepackage{algpseudocode}
\usepackage[noend,algoruled,linesnumbered]{algorithm2e}
\SetKwProg{Fn}{Function}{}{end}
\SetEndCharOfAlgoLine{}
\usepgfplotslibrary{groupplots}
\pgfplotsset{compat=1.9}
%\pgfplotsset{compat=1.12}
\usepackage{multirow}
\makeatletter
\newcommand\footnoteref[1]{\protected@xdef\@thefnmark{\ref{#1}}\@footnotemark}
\makeatother

\usepackage{makecell}

%\usepackage{scrextend}
\usepackage{hyperref}
\usepackage{threeparttable}


\usepackage{array}
\usepackage{colortbl}



\usepackage{etoolbox}
\makeatletter
\patchcmd\@combinedblfloats{\box\@outputbox}{\unvbox\@outputbox}{}{%
  \errmessage{\noexpand\@combinedblfloats could not be patched}%
}%
\makeatother

\usepackage{listings}
\lstdefinelanguage{p4}
{ morekeywords={*,extern_type, attribute, type, method, extern, action, control, void, parser,state, start, transition, extract, select, default, accept, out, in, inout, return},
  sensitive=true,
  morecomment=[l]{//}, % l is for line comment
  morecomment=[s]{/*}{*/}, % s is for start and end delimiter
  morestring=[b]" % defines that strings are enclosed in double quotes
}

\usepackage{fixltx2e}
\usepackage[normalem]{ulem}
\useunder{\uline}{\ul}{}

\hyphenation{pro-gram-ma-ble}
\usetikzlibrary{arrows,shapes.gates.logic.US,shapes.gates.logic.IEC,calc}

%\usepackage[frenchb,english]{babel}
%\usepackage[utf8]{inputenc}
\usepackage{amssymb}


\usepackage{pgfplots}
\usepackage{tikz,tkz-tab}%package pour les tableaux de variations et de signes%
%\usetikzlibrary{babel}
%\usepackage{modules/tikzNetwork}
\usepackage{circuitikz}
\usetikzlibrary{positioning, automata, graphs, trees, fit, arrows, shapes}
\usetikzlibrary{backgrounds,patterns,matrix,calc,shadows,plotmarks, circuits.logic.US}
\usetikzlibrary{decorations}

\usepackage{lipsum}
\usepackage{setspace}
\usepackage{xfrac}
%\usepackage{xcolor}
\usepackage{xcolor}

\usepackage{listings}
\lstset{
	breaklines=true,
	numbers=left,
	xleftmargin=5.0ex,
	columns=fullflexible,
	language=C++,
	morekeywords={*,size_t, auto, decltype, constexpr, array, override},
	float=!ht,
	basicstyle=\scriptsize\ttfamily,
	frame=lines, 
	escapeinside={(*@}{@*)},
	tabsize=4,
                keywordstyle=\color{blue},
                stringstyle=\color{red},
                commentstyle=\color{Green},
                morecomment=[l][\color{magenta}]{\#}
}

%\setlength{\belowcaptionskip}{-6pt}
%\setlength{\abovecaptionskip}{-0pt}

\DeclareSIUnit{\bit}{b}
\DeclareSIUnit{\byte}{B}
\DeclareSIUnit{\update}{Updates}

\newcommand\Hidden{No} % Yes or No

%% Patch Captions
\captionsetup[figure]{labelsep=period,font=footnotesize}
\captionsetup[subfigure]{font=footnotesize}
\captionsetup[table]{justification=centering,labelsep=newline,font={footnotesize,sc}}



\IEEEoverridecommandlockouts

\begin{document}

\bstctlcite{IEEEexample:BSTcontrol}

%
% paper title
% Titles are generally capitalized except for words such as a, an, and, as,
% at, but, by, for, in, nor, of, on, or, the, to and up, which are usually
% not capitalized unless they are the first or last word of the title.
% Linebreaks \\ can be used within to get better formatting as desired.
% Do not put math or special symbols in the title.
\title{Hit the Jackpot! A Match-Action Cache Policer \\for Heterogeneous Programmable Dataplanes}
%\title{A Match-Action Cache Policer for Heterogeneous Programmable Dataplanes}


% author names and affiliations
% use a multiple column layout for up to three different
% affiliations

%\ifthenelse{\equal{\Hidden}{No}}{
%\author{\IEEEauthorblockN{Jeferson Santiago da Silva\dag, Thibaut Stimpfling\dag, Thomas Luinaud\dag, Fran\c{c}ois-Raymond Boyer\dag, J.M. Pierre Langlois\dag and Ludovic Beliveau}
%\IEEEauthorblockA{Polytechnique Montr\'{e}al\dag,\\
%Kaloom Inc.\dag\ddag,
%Montr\'{e}al, Canada\\
%\{jeferson.silva\}@polymtl.ca}}
%}{
%\author{\IEEEauthorblockN{First Author, Second Author and Third Author}
%\IEEEauthorblockA{Authors' Institution, Authors' Country\\
%Authors' email}}
%}


\ifthenelse{\equal{\Hidden}{No}}{
\author{
\IEEEauthorblockN{Jeferson Santiago da Silva\IEEEauthorrefmark{2}\IEEEauthorrefmark{1}\\[0.3cm]Fran\c{c}ois-Raymond Boyer\IEEEauthorrefmark{2}}
\and
\IEEEauthorblockN{Thibaut Stimpfling\IEEEauthorrefmark{2}\\[0.3cm]J.M. Pierre Langlois\IEEEauthorrefmark{2}}
\and
\IEEEauthorblockN{Thomas Luinaud\IEEEauthorrefmark{2}\\[0.3cm]Ludovic Beliveau\IEEEauthorrefmark{3}}
\thanks{
\IEEEauthorrefmark{2}Polytechnique Montr\'{e}al, Montr\'{e}al, Canada.
  
\IEEEauthorrefmark{3}Kaloom Inc., Montr\'{e}al, Canada.

\IEEEauthorrefmark{1}Corresponding author: jeferson.silva@polymtl.ca.
}
}
}{
\author{\IEEEauthorblockN{First Author, Second Author and Third Author}
\IEEEauthorblockA{Authors' Institution, Authors' Country\\
Authors' email}}
}

% conference papers do not typically use \thanks and this command
% is locked out in conference mode. If really needed, such as for
% the acknowledgment of grants, issue a \IEEEoverridecommandlockouts
% after \documentclass

% for over three affiliations, or if they all won't fit within the width
% of the page, use this alternative format:
% 
%\author{\IEEEauthorblockN{Michael Shell\IEEEauthorrefmark{1},
%Homer Simpson\IEEEauthorrefmark{2},
%James Kirk\IEEEauthorrefmark{3}, 
%Montgomery Scott\IEEEauthorrefmark{3} and
%Eldon Tyrell\IEEEauthorrefmark{4}}
%\IEEEauthorblockA{\IEEEauthorrefmark{1}School of Electrical and Computer Engineering\\
%Georgia Institute of Technology,
%Atlanta, Georgia 30332--0250\\ Email: see http://www.michaelshell.org/contact.html}
%\IEEEauthorblockA{\IEEEauthorrefmark{2}Twentieth Century Fox, Springfield, USA\\
%Email: homer@thesimpsons.com}
%\IEEEauthorblockA{\IEEEauthorrefmark{3}Starfleet Academy, San Francisco, California 96678-2391\\
%Telephone: (800) 555--1212, Fax: (888) 555--1212}
%\IEEEauthorblockA{\IEEEauthorrefmark{4}Tyrell Inc., 123 Replicant Street, Los Angeles, California 90210--4321}}




% use for special paper notices
%\IEEEspecialpapernotice{(Invited Paper)}




% make the title area
\maketitle



% As a general rule, do not put math, special symbols or citations
% in the abstract
\begin{abstract}
\lipsum[1-2]
\end{abstract}



% no keywords



% For peer review papers, you can put extra information on the cover
% page as needed:
% \ifCLASSOPTIONpeerreview
% \begin{center} \bfseries EDICS Category: 3-BBND \end{center}
% \fi
%
% For peerreview papers, this IEEEtran command inserts a page break and
% creates the second title. It will be ignored for other modes.
\IEEEpeerreviewmaketitle

\section{Introduction}\label{sec:intro}

The Software-Defined Networking (SDN) paradigm has brought programmability to the once rigid network ecosystem. By allowing both control and data planes to evolve independently, SDN has opened new research avenues in networking, including data plane programming. Notably, the P4 language is a result of the SDN convergence \cite{Bosshart:14}. P4 leverages OpenFlow by allowing network administrators to program their network in a custom fashion. Thanks to P4 and recent programmable dataplanes \cite{Bosshart:13}, they can now deploy custom protocols by reprogramming network switches according to evolving needs, without deploying expensive new hardware. 


However, current data center networks (DCNs) requirements are such \textcolor{red}{[ref]} that even state-of-the-art programmable ASICs \cite{tofino:18} cannot solely meet them. Next generation mobile communication (5G) requirements, for instance, include multi-million active sessions ($>$\SI{5}{\mega\nothing}) at terabit rates and stringently low end-to-end latency ($<$\SI{1}{\milli\second}), possibly running over custom protocols.

Recently, some research has suggested using heterogeneous programmable data planes (HDPs) to alleviate data center network switch bottlenecks \cite{p4eu:18}. Indeed, using complementary and distinct packet forwarding devices increases the overall switch processing capabilities. However, research regarding HDPs is still in its infancy with many questions to be solved, including the development of heterogeneous compilers, mismatched processing capabilities and hardware resources, and match-tables management.

In this work, we address the issue of match-action (M-A) table management in HDPs comprising a high speed programmable ASIC, an FPGA, and a host CPU. To that end, we borrow the cache hierarchy concept of regular computer systems. In our system, a first-level cache is the high-performance but memory limited programmable ASIC. The FPGA plays the role of second-level cache, and finally the CPU is the last-level cache. M-A caching is fairly different from regular CPU cache systems since temporal and spatial data locality are difficult to predict in network systems. 

Thus, we propose a M-A cache policer split into the forwarding devices. We use online traffic hitters to estimate which M-A entries are candidates to be promoted/evicted to/from another cache level. We are inspired in previous works on flow caching \cite{casado:2008,Katta:2014,Pfaff:15} and on heuristic dataplane-based traffic hitters \cite{Sivaraman:17} aiming to maintain line-rate throughput and required cache update rate while minimizing the usage of scarce memory resources available in each device and reducing processing latency.

To the best of our knowledge, our work is the first to consider a M-A cache policer for HDPs. The contributions of this work are as follows: 
\begin{itemize}[noitemsep,topsep=0pt]
	\item an open-source P4-based match-action cache policer for HDPs (\S\ref{sec:method});
	\item a model to estimate the performance and overhead of a M-A cache system in an HDP (\S\ref{sec:model}); and
	%\item a P4-based emulation prototyping platform (\S\ref{sec:emulation}). 
	\item a P4-based emulation platform for rapid performance estimation (\S\ref{sec:emulation}); and 
	\item a prototype comprising a programmable ASIC, an FPGA, and a CPU.
\end{itemize}

%The rest of this paper is organized as follows. Section~\ref{sec:eco_pro} presents the ecosystem this work is inserted into and states the problem that we propose to solve. Section~\ref{sec:related_works} reviews the literature, Section~\ref{sec:method} presents the proposed match-action cache policer, Section~\ref{sec:results} shows the experimental results and discussions, and Section~\ref{sec:conclusion} draws the conclusions.


\section{Network Ecosystem and Problem Statement}\label{sec:eco_pro}

%In this section, we first introduce the ecosystem in which our work is inserted into by presenting terms and definitions used in this paper. Then, we state the problem that this work proposes to solve.

In this section, we first introduce the ecosystem in which our work is inserted into, and then, we state the problem that this work proposes to solve.

\subsection{Network Ecosystem}\label{sec:eco}

In this paper, we consider an SDN switch for a leaf-spine data center network as illustrated in Figure~\ref{fig:high_level_network}. In the figure, solid lines indicate the data plane and dashed lines refer to control plane communication. Dotted lines delimit the cache levels of an HDP. The next paragraphs describe terms and definitions used throughout this text.

\begin{figure}[]
	\centering
	%\begin{tikzpicture}[node distance = 0.5cm, every node/.style={draw}, node font={\footnotesize}]
%%software
%\begin{scope}[every node/.style={draw, chamfered rectangle, chamfered rectangle corners=north east}]
%	\node (globp4) {P4};
%	\node[rectangle, right=of globp4] (compiler) {HDP compiler};
%	\node[below= 0.5cm of compiler] (p4cpu2) { P4};
%	\node[left= of p4cpu2] (p4cpu1) {P4};
%	\node[left= of p4cpu1] (p4fpga1) {P4};
%	\node[right= of p4cpu2] (p4fpga2) {P4};
%	\node[right= of p4fpga2] (p4asic) { P4};
%%%path for the corner of p4 codes
%	\path[draw]  (globp4.before north east) -| (globp4.after north east)
%							  (p4cpu2.before north east) -| (p4cpu2.after north east)
%							  (p4cpu1.before north east) -| (p4cpu1.after north east)
%						      (p4fpga1.before north east) -| (p4fpga1.after north east)
%							  (p4fpga2.before north east) -| (p4fpga2.after north east)
%							  (p4asic.before north east) -| (p4asic.after north east);
%
%%%compilation
%	\begin{scope}[every node/.style={rectangle, draw, align=center}, node distance=0.25cm]
%		\node[below= of p4cpu2] (compcpu2) { CPU\\ comp};
%		\node[below= of p4cpu1] (compcpu1) { CPU\\ comp};
%		\node[below= of p4fpga1] (compfpga1) { FPGA\\ comp};
%		\node[below= of p4fpga2] (compfpga2) { FPGA\\ comp};
%		\node[below= of p4asic] (compasic) { ASIC\\ comp};
%	\end{scope}
%%%processus
%	\begin{scope}[every edge/.style={-Stealth, draw}]
%   \coordinate (midBelowComp) at ($(compiler.south) - (0,0.25)$);
%	\path[draw] (globp4) edge (compiler)
%	    							  (compiler.south) edge (p4cpu2)
%								  ($(compiler.south) - (0.2,0)$) |- ( p4cpu1.north |- midBelowComp) edge (p4cpu1)
%                                  ($(compiler.south) - (0.4,0)$) |- ($( p4fpga1.north |- midBelowComp) + (0,0.1)$) edge (p4fpga1)
%								 ($(compiler.south) + (0.2,0)$) |- ( p4fpga2.north |- midBelowComp) edge (p4fpga2)
%                                  ($(compiler.south) + (0.4,0)$) |- ($( p4asic.north |- midBelowComp) + (0,0.1)$) edge (p4asic)
%								 (p4cpu1) edge (compcpu1)
%								 (p4cpu2) edge (compcpu2)
%								 (p4fpga1) edge (compfpga1)
%								 (p4fpga2) edge (compfpga2)
%								 (p4asic) edge (compasic)
%									;
%	\end{scope}
%
%\end{scope}
%\coordinate (midCPU) at ($(p4cpu1.east |- compcpu1.south) + (0.25cm, -0.5cm)$);
%%%hardware
%\begin{scope}[every node/.style={draw},]
%	\node[left=0.25cm of midCPU, anchor=north east] (cpu1) {CPU\textsubscript{1}};
%	\node[right=0.25cm of midCPU, anchor=north west] (cpu2) {CPU\textsubscript{2}};
%	\node[below=of cpu1, fill=black!20] (fpga1) {FPGA\textsubscript{1}};
%	\node[below=of cpu2] (fpga2) {FPGA\textsubscript{2}};
%	\node[fill=black!20, minimum width=2.25cm, anchor = north] (asic) at ($(midCPU|-fpga2.south) - (0,0.5)$) {ASIC};
%	\node[draw=none,left =0.55 of asic, align=left] (data_label) {};
%	
%\end{scope}
%\node[fit=(cpu1) (cpu2) (fpga1) (fpga2) (asic) (data_label), rounded corners] (platform) {};
%\node[draw=none, anchor=west] at (platform.west |- asic.east) {HDP};
%\node[cloud, draw, aspect=2, inner sep=-2pt, below=0.4cm of asic] (network) {Network};
%\begin{scope}[every edge/.style={Stealth-Stealth, draw}]
%\path (cpu1) edge (cpu2)
%			(cpu1) edge (fpga1)
%			(cpu2) edge (fpga2)
%			(fpga1.south) edge (fpga1.south |- asic.north)
%			(fpga2.south) edge (fpga2.south |- asic.north)
%			;
%\end{scope}
%\begin{scope}[every edge/.style={thick, -Latex, dashed, draw}, every path/.style={thick, dashed}]
%		\path[draw] ($(platform.west |- compfpga1.south) - (0.1,0)$) -- ($(platform.west |- fpga1) - (0.1,0)$)  edge (fpga1)
%								(compfpga2) -- (compfpga2 |- fpga2) edge (fpga2)
%								(compasic) -- (compasic |- asic) edge (asic)
%								(compcpu2) edge (compcpu2.south |- cpu2.north) 
%								(compcpu1) edge (compcpu1.south |- cpu1.north) 
%								;
%	\end{scope}
%\path[Stealth-Stealth, draw] (network) edge (asic.south)
% 							   (network) edge ($(asic.south) + (0.2,0cm)$)
% 							   (network) edge ($(asic.south) - (0.2,0cm)$)
% 							   (network) edge ($(asic.south) + (0.4,0cm)$)
% 							   (network) edge ($(asic.south) - (0.4,0cm)$);
%	\node[draw, anchor = west, fill=black!20](control) at (network.east -| platform.east) { Control-plane};	
%	  \path[densely dashed, Stealth-Stealth, draw] (control) -| ($(platform.south east) - (0.5cm,0)$);
%	\node[fit = (compfpga1) (compasic) (compiler), draw, dash dot dot, thick] (comp) {};
%	\node[draw=none, below left=0 and 0 of comp.north east, align=left] { Compilation\\ Process};
%	\end{tikzpicture}
%	
%	

\tikzset{%
  block/.style    = {draw, very thick, rectangle, minimum height = 2em,
    minimum width = 3em},
  sum/.style      = {draw, circle, node distance = 2cm}, % Adder
  input/.style    = {coordinate}, % Input
  output/.style   = {coordinate} % Output
}
% Defining string as labels of certain blocks.
%\newcommand{\suma}{\Large$+$}
%\newcommand{\inte}{$\displaystyle \int$}
%\newcommand{\derv}{\huge$\frac{d}{dt}$}

\tikzstyle{block} = [draw, rectangle, 
    minimum height=1em, minimum width=3em]
\tikzstyle{sum} = [draw, circle, node distance=1cm]
\tikzstyle{input} = [coordinate]
\tikzstyle{output} = [coordinate]
\tikzstyle{pinstyle} = [pin edge={to-,thin,black}]

\begin{tikzpicture}[auto, node distance=1cm,>=latex']
    \node [input, name=input] {};


    \node [block, below of=input, node distance=.85cm] (cpu) {\small\textbf{PDD\textsubscript{N}}};

    \node [block, right=2 cm of cpu, node distance=.85cm] (cpu1) {\small\textbf{PDD\textsubscript{N}}};
    
\draw[dotted, semithick, name=mm] ($(cpu.north west)+(-2.5cm,+.2cm)$) -- ($(cpu1.north east)+(1cm,+.2cm)$);

    \node [draw=none] at ($(cpu.north west)+(-1.5cm,+.55cm)$) (main) {\scriptsize{\textit{Main Memory}}};


\draw[dotted, semithick] ($(cpu.south west)+(-2.5cm,-.2cm)$) -- ($(cpu1.south east)+(1cm,-.2cm)$);

    \node [draw=none] at ($(cpu.west)+(-1.5cm,-.0cm)$) (hdp) {\scriptsize{\textit{Cache Level N}}};

    \node [block, below of=cpu, node distance=1.75cm] (fpga) {\small\textbf{PDD\textsubscript{2}}};

    \node [draw=none] at ($(fpga.north)+(.0cm,.75cm)$) (dotspdd) {\textbf{$\vdots$}};

    \node [draw=none] at ($(dotspdd)+(1.5cm,-.1cm)$) (dots) {\textbf{$\dots$}};

    \node [draw=none] at ($(dots)+(0cm,.75cm)$) (dotp) {\textbf{$\dots$}};

    \node [draw=none] at ($(dots)+(0cm,-.75cm)$) (dotm) {\textbf{$\dots$}};

    \node [draw=none] at ($(dotm)+(0cm,-1.0cm)$) (dotmm) {\textbf{$\dots$}};


    \node [block, below of=cpu1, node distance=1.75cm] (fpga1) {\small\textbf{PDD\textsubscript{2}}};

    \node [draw=none] at ($(fpga1.north)+(.0cm,.75cm)$) (dotspdd1) {\textbf{$\vdots$}};


\draw[dotted, semithick] ($(fpga.north west)+(-2.5cm,+.2cm)$) -- ($(fpga1.north east)+(1cm,+.2cm)$);

\draw[dotted, semithick] ($(fpga.south west)+(-2.5cm,-.2cm)$) -- ($(fpga1.south east)+(1cm,-.2cm)$);


    \node [draw=none] at ($(fpga.west)+(-1.5cm,-.0cm)$) (ll) {\scriptsize{\textit{Cache Level 2}}};

    \node [draw=none] at ($(ll.north)+(.0cm,.75cm)$) (dotscl) {\textbf{$\vdots$}};

    \node [block, below of=fpga, node distance=1.0cm] (asic) {\small\textbf{PDD\textsubscript{1}}};

    \node [block, below of=fpga1, node distance=1.0cm] (asic1) {\small\textbf{PDD\textsubscript{1}}};



    \node [draw=none] at ($(asic.west)+(-1.5cm,-.0cm)$) (hdp) {\scriptsize{\textit{Cache Level 1}}};

    \node [block,node distance=.85cm, minimum width=7em]  at ($(cpu.north)+(1.5cm,.70cm)$) (controller) {\textbf{Controller}};


	\node[cloud, draw, aspect=2, inner sep=-2pt,minimum width=5em, minimum height=3em] at ($(asic.south)+(1.5cm,-.75cm)$) (network) {\textbf{Network}};

    \node [output, below of=network] (output) {};

    % Once the nodes are placed, connecting them is easy. 
    %%%\draw [thick,<->] (fpga) -- node {} (cpu);
    %%%\draw [thick,<->] (fpga1) -- node {} (cpu1);

    \draw [thick,<->] (cpu) -- node {} ($(dotspdd.center)+(0.0cm,.10cm)$);
    \draw [thick,<->] (fpga) -- node {} ($(dotspdd.south)+(0.0cm,.09cm)$);

    \draw [thick,<->] (cpu1) -- node {} ($(dotspdd1.center)+(0.0cm,.10cm)$);
    \draw [thick,<->] (fpga1) -- node {} ($(dotspdd1.south)+(0.0cm,.09cm)$);
    
    \draw [dashed, <->] (fpga.west) --+(-0.25cm,0) |- node {} (asic);
    \draw [dashed, <->] (cpu.west) --+(-0.25cm,0) |- node {} (fpga);
    \draw [thick,<->] (fpga) -- node {} (asic);
    \draw [very thick,<->] (asic) |- node {} (network);

    \draw [dashed, <->] (fpga1.east) --+(0.25cm,0) |- node {} (asic1);
    \draw [dashed, <->] (cpu1.east) --+(0.25cm,0) |- node {} (fpga1);
    \draw [thick,<->] (fpga1) -- node {} (asic1);
    \draw [very thick,<->] (asic1) |- node {} (network);

    \draw [dashed, <->] (cpu.west)  --+(-0.25cm,0) |- node {} (controller.west);
    \draw [dashed, <->] (cpu1.east) --+(0.25cm,0) |- node {}(controller.east);


    \draw []($(cpu1.north)+(-.65cm,.15cm)$) rectangle ($(asic1.south)+(0.9cm,-.30cm)$);
    
    \draw []($(cpu.north)+(-0.9cm,.15cm)$) rectangle ($(asic.south)+(0.65cm,-.30cm)$);
    
    \node [draw=none] at ($(asic.south west)+(-.075cm,-.15cm)$) (hdp) {\footnotesize{HDP\textsubscript{1}}};
    \node [draw=none] at ($(asic1.south east)+(.075cm,-.15cm)$) (hdp) {\footnotesize{HDP\textsubscript{N}}};

\end{tikzpicture}

	\caption{Reference DCN system}
	\label{fig:high_level_network}
\end{figure}

\begin{description}[noitemsep,topsep=0pt,labelwidth=0pt,leftmargin=0pt]
%\item[Packet Forwarding Device:] a device capable of forwarding packets between networks.
\item[Heterogeneous Programmable Dataplane (HDP):] an SDN switch composed of three or more distinct packet forwarding devices programmed through a packet processing language \cite{Bosshart:13}.
\item[Controller:] a centralized control plane entity responsible for managing SDN programmable dataplanes.
\item[Host Switch CPU:] a local switch processor responsible for installing forwarding rules into packet forwarding devices according to instructions received from a controller through a well-known interface \cite{of:14, p4_runtime:18}. The number and types of forwarding devices in a programmable pipeline is abstracted from the controller by the host switch CPU.
\end{description}

Table~\ref{tab:requirements} list the requirements we consider for a today's data center HDP switch. Commercial programmables switches have far crossed the \SI{}{\tera\bit/\second} barrier \cite{tofino:18}. Multi \SI{100}{\giga\bit/\second} FPGA NICs are also commercially available \cite{XilinxFPGA:18,IntelFPGA:18}. Standard datacenter racks hold up to 40 servers on which hundreds of multi-tenant containers ($>$1000) \textcolor{red}{[ref]} can run. Each of these containers have dozens of active flows. In summary, one expects at least $40\times1000\times100 =$~\SI{4}{\mega\nothing} active flows per top-of-rack (leaf) switch \textcolor{red}{[ref]}. 


\begin{table}[]
\centering
\caption{HDP system requirements}
\label{tab:requirements}
\begin{tabular}{|l|l|}
\hline
\textbf{Parameter}      & \textbf{Requirement}         \\\hline
Aggregate bandwidth            & \SI{6}{\tera\bit/\second}    \\
ASIC-FPGA bandwidth            & \SI{200}{\giga\bit/\second}  \\
Concurrent flows               & \SI{5}{\mega\nothing}        \\
M-A table update rate          & \SI{5}{\kilo\update/\second} \\\hline
\end{tabular}
\end{table}

\subsection{Problem Statement}\label{problem}

Heterogeneous programmable dataplanes are required to achieve the current needs of data center applications. However, devices making up an HDP differ in processing capabilities in their most varied forms. A programmable ASIC ($>$\SI{6}{\tera\bit/\second}) has high throughput at the expense of limited programmability and memory capacity ($<$\SI{100}{\mega\bit}), while an FPGA adds reconfigurability and memory abundance ($>$\SI{8}{\giga\byte}) at the cost of reduced performance ($<$\SI{1}{\tera\bit/\second}).

The controller manages switches, installs forwarding rules, and collects statistics. However, this controller is unaware of the switch architecture. Thus, the switch host CPU splits these rules into each device that make up an HDP. To maximize efficiency, the host CPU exploits the characteristics of each device for forwarding rules insertion. In this way, frequently matched rules shall be installed into a high performance programmable ASIC while infrequently ones may be placed in a memory abundant, internally or externally, FPGA device. 

However, entirely managing M-A cache policy at such high data rates in a Von-Neumann CPU is impractical. Thus, cache replacement algorithms need to be implemented in dataplane devices to alleviate the burden on the CPU. The switch host CPU has only to periodically monitor candidate flows for cache migration. These flow candidates can be either evicted from high performance devices or promoted from the memory abundant to the high-speed device.

\section{Related Work}\label{sec:related_works}

%P4 compilers for different targets. Why ASICs and FPGAs? Programmable, unmatched performance characteristics,...
%State-of-the-art FPGAs: very-high memory bandwidth (internal, external), increased performance, programmability...
%
%Movement towards HDP. Bandwidth requirements and large number of flows.
%
%Traditional caching systems are not good match for packet processing in general. Packet processing has low temporal and spatial locality. Present the parallel CPU-GPU systems or even FPGA-CPU systems.
%
%Traffic-hitters have been mainly used for monitoring tasks.
%
%M-A caching as in \cite{Grigoryan:18} is infeasible in a high-performance switch.
%
%
%\subsection{Cache Policy Mechanisms}\label{sec:related_works:cache}
%
%LRU, LFU?? Consensus on LRU
%
%LFRU


%\subsection{Flow Caching}\label{sec:flow_caching}

\textit{Flow caching} has been studied since the early times in flow-based networking. Casado \textit{et al.} \cite{casado:2008} remarked in 2008 that a hardware-based SDN switch must achieve over 99\% hit-ratio to avoid system bottlenecks due to software interaction.

Since then, flow caching has been explored for both hardware and software solutions. Katta \textit{et. al} \cite{Katta:2014,Katta:2016} have addressed the issue of limited TCAM resources in hardware switches by proposing a hybrid hardware-software switch to exploit memory-abundant CPUs. The cache policy algorithm, however, is performed offline. From the software side, the Open vSwitch (OVS) has employed flow caching from its inception \cite{Pfaff:15}. In OVS, the flow cache is split in two levels, microflow and megaflow. The microflow caches at fine granularity for long lasting connections while the megaflow, at coarser granularity, takes care of short-lived flows.

%The microflow caches the hash of specific header fields to identify recurring connections. While microflow caching performs well for long lasting connections it suffers when a large number of short-lived ones arrive at the switch. To deal with it, the megaflow cache implements a generic matching following the OpenFlow matching style, however, with no matching priorities. To minimize prefix ambiguity, OVS caches only disjoint megaflows. 

Grigoryan and Liu have proposed a programmable FIB caching architecture \cite{Grigoryan:18}. They were inspired by the heavy-hitter implementation of \cite{Sivaraman:17} to detect and evict infrequent M-A TCAM entries. However, their approach requires data-plane based learning for cache replacement and it assumes that the switch is able to deal with non-deterministic lookup time, which can compromise performance due to pipeline stalls. 

Zhang \textit{et al.} have presented B-cache, a behavior-level cache for programmable dataplanes \cite{Zhang:2018}. Similarly to Grigoryan \textit{et al.} \cite{Grigoryan:18}, the authors exploit heavy-hitters to identify hot behavior in programmable dataplanes, which in turn could be cached. Similarly, this works is infeasible in current homogeneous high-performance switches since it breaks the streaming flow in the pipeline.

Kim \textit{et al.} have proposed extending the memory capacity in programmable switches by borrowing memory resources from RDMA-capable servers in data centers \cite{Kim:2018}. However, the achieved latency can be in the order of microseconds and the switch does not consider any cache policy mechanism.

%\subsection{Traffic Hitters}\label{sec:related_works:hitters}

\textit{Traffic Hitters} have mainly been used for network monitoring and management with the goal of identifying hot network behavior, and potential attacks. Realistic implementations of traffic hitters have mainly used sketch algorithms and their variations \cite{Cormode:05,Cormode:08,Metwally:05} in order to increase memory efficiency. Liu \textit{et al. } \cite{Liu:16} have deployed network-wide flow monitoring. Sivaraman \textit{et al. } \cite{Sivaraman:17} have adapted the classical finding the top-k element problem \cite{Metwally:05} to map it efficiently to programmable switches \cite{Bosshart:13}. FPGA-based deployments have also been proposed \cite{Tong:13,Tong:16,Zazo:17}.

\section{Proposed Method}\label{sec:method}

\subsection{High-Level Overview}\label{sec:method:overview}

The proposed heterogeneous M-A cache policer is described in Algorithm~\ref{alg:cache_pol}. Candidates for flow migration are online detected by light/heavy hitters implemented on the ASIC and FPGA. For this, we are inspired by the work of Sivaraman \textit{et al.} \cite{Sivaraman:17} due its high performance and low memory footprint.

\begin{algorithm}[!t]
\caption{Match-action cache policer}
\label{alg:cache_pol}
\SetInd{0.1em}{.9em}
\SetAlgoLined
\footnotesize
\SetKwProg{procedure}{Procedure}{}{end}
\SetKwFunction{maCachePolicer}{maCachePolicer}
\SetKwFunction{evalCacheUpdate}{evalCacheUpdate}
\SetKwFunction{installAsicRules}{installAsicRules}%
\SetKwFunction{detectFpgaCand}{detectFpgaCand}%
\SetKwFunction{detectAsicVictims}{detectAsicVictims}%
\SetKwFunction{installFpgaRules}{installFpgaRules}%
\SetKwFunction{getCpuTimer}{getCpuTimer}%

\SetKw{In}{in}%
\SetKw{Not}{not}%
\SetKw{Or}{or}%
\SetKw{Continue}{continue}%
\SetKw{True}{true}%
\SetKwInOut{Input}{input}
\SetKwInOut{Output}{output}
\SetKwRepeat{Do}{while}{do}
\Input{List of new flow entries to be added}
\Input{List of flow entries to be removed}
\Input{List of flow entries to be updated}
%\Output{Optimized balanced graph}
%\KwData{A node is a data structure that has pointers to $successors$/$predecessors$ and methods to add/remove them. Also, a node has a $level$, representing the graph level and it is unassigned at the beginning.}
\procedure{\maCachePolicer{newCtrlEntries, delCtrlEntries, updateCtrlEntries}}{

	\While{\True}{
		\tcc{CPU timer for updating entries}
		$cpuUpdateKernel = \getCpuTimer()$\label{alg:cache_pol:cpu_timer}
	
		\tcc{Evaluate FPGA frequent matches and update candidates list}
		$fpgaCand = \detectFpgaCand()$\label{alg:cache_pol:fpga_cand}

		\tcc{Evaluate ASIC infrequent matches and update victims list}
		$asicViticms = \detectAsicVictims()$\label{alg:cache_pol:asic_victims}
	
		\tcc{Timely CPU update kernel}
		\If{$cpuUpdateKernel$}{
			\If{$newCtrlEntries \neq \emptyset$ \Or $delCtrlEntries \neq \emptyset$ \Or $updateCtrlEntries \neq \emptyset$}{
				\If{$delCtrlEntries \in ASIC$ \Or $updateCtrlEntries \in ASIC$}{
				\tcc{Update/remove controller rules if already installed in the ASIC}
				$\installAsicRules(\emptyset, delCtrlEntries,$ \\\qquad $updateCtrlEntries)$\label{alg:cache_pol:delete_asic}
				}
				\tcc{Controller rules are always installed into the FPGA}
				$\installFpgaRules(newCtrlEntries,$ \\\qquad $delCtrlEntries, updateCtrlEntries)$\label{alg:cache_pol:install_fpga}
			}\Else{
				\tcc{CPU evaluates rules migration based on flow activity and system aggregated bandwidth}
				$newEntries, delEntries = \evalCacheUpdate(fpgaCand, asicViticms)$\label{alg:cache_pol:eval_migration}\\
				$\installAsicRules(newEntries, delEntries, \emptyset)$\label{alg:cache_pol:install_asic}
			}
		}
	}
}
\end{algorithm}

The \texttt{maCachePolicer} procedure shown in Algorithm~\ref{alg:cache_pol} has two controller-managed inputs: $newCtrlEntries$, $delCtrlEntries$, and $updateCtrlEntries$. These inputs control insertion, deletion, and updating of rules in the switch. In our work, new rules are always inserted, deleted, and updated directly in the the FPGA, as shown by line~\ref{alg:cache_pol:install_fpga}. New entries mat eventually migrate to the ASIC according to the dynamics of the cache replacement algorithm. Deleting or updating entries is executed in the ASIC, as in line~\ref{alg:cache_pol:delete_asic}.

%TODO: Consider remove/update entries in the ASIC.

\texttt{detectFpgaCand} and \texttt{detectAsicVictims} are implemented in the dataplane. \texttt{detectFpgaCand} (line~\ref{alg:cache_pol:fpga_cand}) implements a heavy hitter in the FPGA to detect frequent matched flows, and therefore, candidates for flow migration. \texttt{detectAsicVictims} (line~\ref{alg:cache_pol:asic_victims}) is a light hitter detector in the ASIC that identifies candidates for cache eviction.

\texttt{evalCacheUpdate} and \texttt{installAsicRules} run in the host CPU a timely-fashion. \texttt{evalCacheUpdate} (line~\ref{alg:cache_pol:eval_migration}) collects flows activity information from both FPGA and ASIC and determines which rules will be inserted/removed into/from the ASIC. Flow information is weighted by the aggregated bandwidth of each device in order to correct estimate flow rates, that are used by the \texttt{installAsicRules} (line~\ref{alg:cache_pol:install_asic}) procedure to update the ASIC M-A tables.

The next subsections give more implementation details regarding the traffic hitters implementation in both ASIC and FPGA devices and the host CPU procedure for rules installing. 

\subsection{Detecting Candidates for Flow Migration}\label{sec:method:detection}

Need to design a minimal heavy hitter implementation to match memory and logic resources.

\subsection{Collecting Candidates and Update Procedure}\label{sec:method:update}

\subsection{Limitations of the Proposed Method}\label{sec:method:limitations}

In this work we are interested in detecting possible candidates for flow migration entirely in the dataplane. This is due high-speed links expected in data center networks and the fast changing nature of data center network traffic; therefore, a slow control plane interaction is undesired. However, candidates migration detection in the dataplane can only be precisely detected for exact match rules due to ambiguity in ternary and LPM match rules. For example, let us consider a case where a low priority LPM rule is frequently matched in a low cache level and would therefore be a candidate for flow migration. Using our method, this rule is moved to a higher cache level as expected. Now, a high priority rule belonging to the same prefix arriving to the HDP switch will match in the high cache level. However, a specific rule installed in a lower cache level for the same prefix will be hidden, leading thus to a possibly wrong forwarding decision.   
Such cache ambiguity is a known problem and it has been reported and addressed in earlier works \cite{Degermark:1997,Katta:2014}.

\section{Preliminary Results}\label{sec:results}

\subsection{Performance Model}\label{sec:model}

\subsection{Experimental Setup and Dataset}\label{sec:results:setup}

To validate our proposed method, we have developed a simulation prototype in P4\textsubscript{16} running on the behavioral model (bmv2)\footnote{\url{https://github.com/p4lang/behavioral-model}} to emulate both ASIC and FPGA components of an HDP. P4Runtime has been used to interact with the emulated HDP. To allow reproducibility, all source codes are open\footnote{\url{https://github.com/hidden_for_blind_review}}.

For this work, we have collected data center traces produced by Facebook Altoona Data Center in 2015 \cite{Roy:15}. The traces include data from three different data center clusters: database, web, and Hadoop servers.

\subsection{P4-based Emulation Platform}\label{sec:emulation}


\subsection{Discussion}\label{sec:results:discussion}


\section{Conclusion}\label{sec:conclusion}






\balance

\bibliographystyle{IEEEtran}
\bibliography{sample-bibliography} 

%\section{Introduction}
%% no \IEEEPARstart
%This demo file is intended to serve as a ``starter file''
%for IEEE conference papers produced under \LaTeX\ using
%IEEEtran.cls version 1.8b and later.
%% You must have at least 2 lines in the paragraph with the drop letter
%% (should never be an issue)
%I wish you the best of success.
%
%\hfill mds
% 
%\hfill August 26, 2015
%
%\subsection{Subsection Heading Here}
%Subsection text here.
%
%
%\subsubsection{Subsubsection Heading Here}
%Subsubsection text here.
%
%
%% An example of a floating figure using the graphicx package.
%% Note that \label must occur AFTER (or within) \caption.
%% For figures, \caption should occur after the \includegraphics.
%% Note that IEEEtran v1.7 and later has special internal code that
%% is designed to preserve the operation of \label within \caption
%% even when the captionsoff option is in effect. However, because
%% of issues like this, it may be the safest practice to put all your
%% \label just after \caption rather than within \caption{}.
%%
%% Reminder: the "draftcls" or "draftclsnofoot", not "draft", class
%% option should be used if it is desired that the figures are to be
%% displayed while in draft mode.
%%
%%\begin{figure}[!t]
%%\centering
%%\includegraphics[width=2.5in]{myfigure}
%% where an .eps filename suffix will be assumed under latex, 
%% and a .pdf suffix will be assumed for pdflatex; or what has been declared
%% via \DeclareGraphicsExtensions.
%%\caption{Simulation results for the network.}
%%\label{fig_sim}
%%\end{figure}
%
%% Note that the IEEE typically puts floats only at the top, even when this
%% results in a large percentage of a column being occupied by floats.
%
%
%% An example of a double column floating figure using two subfigures.
%% (The subfig.sty package must be loaded for this to work.)
%% The subfigure \label commands are set within each subfloat command,
%% and the \label for the overall figure must come after \caption.
%% \hfil is used as a separator to get equal spacing.
%% Watch out that the combined width of all the subfigures on a 
%% line do not exceed the text width or a line break will occur.
%%
%%\begin{figure*}[!t]
%%\centering
%%\subfloat[Case I]{\includegraphics[width=2.5in]{box}%
%%\label{fig_first_case}}
%%\hfil
%%\subfloat[Case II]{\includegraphics[width=2.5in]{box}%
%%\label{fig_second_case}}
%%\caption{Simulation results for the network.}
%%\label{fig_sim}
%%\end{figure*}
%%
%% Note that often IEEE papers with subfigures do not employ subfigure
%% captions (using the optional argument to \subfloat[]), but instead will
%% reference/describe all of them (a), (b), etc., within the main caption.
%% Be aware that for subfig.sty to generate the (a), (b), etc., subfigure
%% labels, the optional argument to \subfloat must be present. If a
%% subcaption is not desired, just leave its contents blank,
%% e.g., \subfloat[].
%
%
%% An example of a floating table. Note that, for IEEE style tables, the
%% \caption command should come BEFORE the table and, given that table
%% captions serve much like titles, are usually capitalized except for words
%% such as a, an, and, as, at, but, by, for, in, nor, of, on, or, the, to
%% and up, which are usually not capitalized unless they are the first or
%% last word of the caption. Table text will default to \footnotesize as
%% the IEEE normally uses this smaller font for tables.
%% The \label must come after \caption as always.
%%
%%\begin{table}[!t]
%%% increase table row spacing, adjust to taste
%%\renewcommand{\arraystretch}{1.3}
%% if using array.sty, it might be a good idea to tweak the value of
%% \extrarowheight as needed to properly center the text within the cells
%%\caption{An Example of a Table}
%%\label{table_example}
%%\centering
%%% Some packages, such as MDW tools, offer better commands for making tables
%%% than the plain LaTeX2e tabular which is used here.
%%\begin{tabular}{|c||c|}
%%\hline
%%One & Two\\
%%\hline
%%Three & Four\\
%%\hline
%%\end{tabular}
%%\end{table}
%
%
%% Note that the IEEE does not put floats in the very first column
%% - or typically anywhere on the first page for that matter. Also,
%% in-text middle ("here") positioning is typically not used, but it
%% is allowed and encouraged for Computer Society conferences (but
%% not Computer Society journals). Most IEEE journals/conferences use
%% top floats exclusively. 
%% Note that, LaTeX2e, unlike IEEE journals/conferences, places
%% footnotes above bottom floats. This can be corrected via the
%% \fnbelowfloat command of the stfloats package.
%
%
%
%
%\section{Conclusion}
%The conclusion goes here.
%
%
%
%
%% conference papers do not normally have an appendix
%
%
%% use section* for acknowledgment
%\section*{Acknowledgment}
%
%
%The authors would like to thank...
%
%
%
%
%
%% trigger a \newpage just before the given reference
%% number - used to balance the columns on the last page
%% adjust value as needed - may need to be readjusted if
%% the document is modified later
%%\IEEEtriggeratref{8}
%% The "triggered" command can be changed if desired:
%%\IEEEtriggercmd{\enlargethispage{-5in}}
%
%% references section
%
%% can use a bibliography generated by BibTeX as a .bbl file
%% BibTeX documentation can be easily obtained at:
%% http://mirror.ctan.org/biblio/bibtex/contrib/doc/
%% The IEEEtran BibTeX style support page is at:
%% http://www.michaelshell.org/tex/ieeetran/bibtex/
%%\bibliographystyle{IEEEtran}
%% argument is your BibTeX string definitions and bibliography database(s)
%%\bibliography{IEEEabrv,../bib/paper}
%%
%% <OR> manually copy in the resultant .bbl file
%% set second argument of \begin to the number of references
%% (used to reserve space for the reference number labels box)
%\begin{thebibliography}{1}
%
%\bibitem{IEEEhowto:kopka}
%H.~Kopka and P.~W. Daly, \emph{A Guide to \LaTeX}, 3rd~ed.\hskip 1em plus
%  0.5em minus 0.4em\relax Harlow, England: Addison-Wesley, 1999.
%
%\end{thebibliography}
%
%
%
%
%% that's all folks


\end{document}


